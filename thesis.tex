\documentclass[12pt, oneside]{book}
\usepackage[slovak]{babel}
\usepackage[utf8]{inputenc} %encoding, resp. format / encoding pozrieť
\usepackage[a4paper,left=30mm,right=20mm,top=25mm,bottom=25mm,bindingoffset=6mm]{geometry}
\usepackage{graphicx}
\usepackage{titlesec}
\usepackage{tabularx}
\usepackage{multirow}
\usepackage{pdfpages}
\usepackage{etoolbox}
\usepackage{pgfplots}
\usepackage{pgfplotstable}
% \pgfplotstableread[col sep=comma,header=false]{
% Rozhodne ano,53.9
% Skor ano,40.6
% Mierne,3.9
% Skor nie,1.6
% Vobec nie,0
% }\data
% \pgfplotsset{
% width=15cm,
% percentage plot/.style={
%     point meta=explicit,
% every node near coord/.append style={
%     align=center,
%     text width=1cm
% },
%     nodes near coords={
%     \pgfmathtruncatemacro\iszero{\originalvalue==0}
%     \ifnum\iszero=0
%         \pgfmathprintnumber{\originalvalue}$\,\%$\\ \pgfmathprintnumber[fixed zerofill,precision=1]{\pgfplotspointmeta}
%     \fi},
% nodes near coords align=vertical,
%     yticklabel=\pgfmathprintnumber{\tick}\,$\%$,
%     ymin=0,
%     ymax=100,
%     enlarge y limits={upper,value=0},
% visualization depends on={y \as \originalvalue}
% },
% percentage series/.style={
%     table/y expr=\thisrow{#1},table/meta=#1
% }
% }

\makeatletter
\newcommand*\suppresschapternumber{%
  \let\@makechapterhead\@makeschapterhead
  \patchcmd{\@chapter}
    {\protect\numberline{\thechapter}}
    {}
    {}{}%
}
\newcommand*\removedotbetweenchapterandsection{%
  \renewcommand\thesection{\thechapter\@arabic\c@section}%
}
\makeatother

\usepackage{scrwfile}

\TOCclone[Dodatok]{toc}{atoc}
\addtocontents{atoc}{\protect\value{tocdepth}=-1}
\newcommand\listofappendices{\listofatoc}

\newcommand*\savedtocdepth{}
\AtBeginDocument{%
  \edef\savedtocdepth{\the\value{tocdepth}}%
}

\let\originalappendix\appendix
\renewcommand\appendix{%
  \originalappendix
  \cleardoublepage
  \addcontentsline{toc}{chapter}{\appendixname}%
  \addtocontents{toc}{\protect\value{tocdepth}=-1}%
  \addtocontents{atoc}{\protect\value{tocdepth}=\savedtocdepth}%
}
\renewcommand{\appendixname}{Príloha}

\usepackage{listings}
%New colors defined below
\definecolor{codegreen}{rgb}{0,0.6,0}
\definecolor{codegray}{rgb}{0.5,0.5,0.5}
\definecolor{codepurple}{rgb}{0.58,0,0.82}
\definecolor{backcolour}{rgb}{0.95,0.95,0.92}
\definecolor{javadocblue}{rgb}{0.25,0.35,0.75} % javadoc

\renewcommand{\lstlistingname}{Algoritmus}% Listing -> Algorithm
\renewcommand{\lstlistlistingname}{Zoznam algoritmov}% List of Listings -> List of Algorithms
\newcommand{\lstnumberautorefname}{Alg.}

%Code listing style named "mystyle"
\lstdefinelanguage{myLang}{%
    alsodigit={-},    % also: alsoletter
    otherkeywords={checkWriteStoragePermission},
    keywords={checkWriteStoragePermission()},keywordstyle=\color{red}}

\lstdefinestyle{mystyle}{
  backgroundcolor=\color{backcolour},   commentstyle=\color{codegreen},
  keywordstyle=\color{javadocblue},
  numberstyle=\tiny\color{black},
  stringstyle=\color{codepurple},
  basicstyle=\ttfamily\footnotesize,
  breakatwhitespace=false,         
  breaklines=true,                 
  captionpos=b,                    
  keepspaces=true,                 
  numbers=left,                    
  numbersep=5pt,                  
  showspaces=false,                
  showstringspaces=false,
  showtabs=false,                  
  tabsize=2
}

%"mystyle" code listing set
\lstset{style=mystyle}

\usepackage{fancyhdr}
\renewcommand{\headrulewidth}{0pt}

\apptocmd{\thebibliography}{\csname phantomsection\endcsname\addcontentsline{toc}{chapter}{\bibname}}{}{}
\usepackage[nottoc,notlot,notlof]{tocbibind}

\usepackage[font=small,format=plain,labelfont=bf,up,textfont=normal,up]{caption}

\renewcommand*{\figureautorefname}{Obr.}
\renewcommand{\baselinestretch}{1.3}

\titleformat*{\section}{\LARGE\bfseries}
\titleformat*{\subsection}{\Large\bfseries}
\titleformat*{\subsubsection}{\large\bfseries}
\titleformat*{\paragraph}{\large\bfseries}
\titleformat*{\subparagraph}{\large\bfseries}
\usepackage{cite}
\usepackage[export]{adjustbox}


\usepackage{mathptmx}
\usepackage{titlesec}
\usepackage{appendix}
\usepackage{titletoc}
\usepackage{minitoc}
\usepackage{tocloft}

\usepackage[square,numbers]{natbib}
\bibliographystyle{abbrvnat}
\usepackage{hyperref}
\usepackage{float} % Allows putting an [H] in \begin{figumie} to specify the exact location of the figure
\usepackage{wrapfig} % Allows in-line images such as the example fish picture

\newcolumntype{b}{X}
\newcolumntype{s}{>{\hsize=.5\hsize}X}

% Turn on the style
\pagestyle{fancyplain}
% Clear the header and footer
\fancyhf{}

\fancyfoot{}
% Set the right side of the footer to be the page number
\fancyfoot[R]{\thepage}

\usepackage{imakeidx} %imakeidx .
\makeindex
\begin{document}


\includepdf[]{includes/file (1).pdf}

\includepdf[]{includes/file.pdf}

\includepdf[]{includes/zadanie.pdf}

\includepdf[]{includes/suhrn.pdf}

\includepdf[]{includes/abstract.pdf}

\includepdf[]{includes/vyhlasenie.pdf}

\includepdf[]{includes/podakovanie.pdf}

\addtocontents{toc}{\protect\thispagestyle{empty}}
\tableofcontents
\thispagestyle{empty}

\newpage
\listoffigures{\protect\thispagestyle{empty}}
\listoftables{\protect\thispagestyle{empty}}
\newpage
\lstlistoflistings{\protect\thispagestyle{empty}}


\newpage

\chapter*{Zoznam skratiek a značiek}
\thispagestyle{empty}

API - Application Programming Interface (Rozhranie pre programovanie aplikácií)
\\
OS - operačný systém
\\
EGL - Embedded-System Graphics Library (Grafická knižnica zabudovaného systému)
\\



\newpage


\chapter*{Úvod}
\addcontentsline{toc}{chapter}{Úvod}

\hspace{15pt} V poslednej dobe je čoraz viac mobilných zariadení s operačným systémom Android. Mobilné technológie a ich vývoj vzrástli vysoko, v čom sa to preukázalo aj na našich životoch. Dnešné mobilné zariadenie môže slúžiť na rôzne účely ako zaznamenávanie videa, email, prehrávanie videa a hudby, prístup na sociálne siete ako Messenger, WhatsApp a Instagram. Mobilné zariadenia sa v poslednej dobe najviac využívajú pre zaznamenávanie videa a prehrávanie médií. Následne video je možné spracovať rôznymi spôsobmi ako strihanie alebo pridanie hudby na pozadie. Ľudia si následne hľadajú aplikácie, pomocou ktorých je možné vykonať tieto spôsoby spracovania, ktoré boli vyššie spomenuté. 

Pri spracovaní videa je potrebné, aby proces fungoval efektívnym spôsobom, čo znamená, aby bol schopný používateľovi ušetriť čas a uľahčiť ovládanie aplikácie. Začiatok spracovania je možné uskutočniť spôsobmi ako výber videa z úložného priestoru a zaznamenávanie videa. Avšak výber videa v rôznych aplikáciách je komplikovaný a výber je väčšinou mimo aplikácie. Pri zaznamenávaní videa by sa mala prehrávať hudba na pozadí, avšak tento spôsob je treba vyriešiť, aby nastavenie hudby alebo ovládanie kamery bolo ľahké a intuitívne. Náhľad videa je uskutočnený ihneď po ukončení zaznamenávania alebo orezávania videa. Problém je, že náhľad videa musí byť uskutočnené počas toho keď ešte video nie je uložené v zariadení. Preto hlavným cieľom našej bakalárskej práce bude vytvoriť aplikáciu pre efektívne spracovanie videa s intuitívnym dizajnom a jednoduchým ovládaním aplikácie. Taktiež spracovanie videa by malo byť kompatibilné pre rôzne veľkosti obrazovky a nemalo by mať záťaž na batériu.

V prvej kapitole sa venujeme detailnému opisu knižníc, pomocou ktorých je možné implementovať komponenty potrebné pre našu aplikáciu. Opísali sme potrebné funkcie a metódy, ktoré dané knižnice poskytujú. Následne sme analyzovali aplikácie podobne tej našej problematike. Opísali sme ich výhody, nevýhody a v čom bude naša aplikácia odlišná.

V druhej kapitole je poskytnutý detailný opis technológií, ktoré sme používali pri vývoji aplikácie. Následne opisujeme aké by mala mať požiadavky naša aplikácia. Ďalšie podkapitoly poskytujú opis hlavných implementácií. V poslednom rade sme sa venujeme vyhodnotení testov, kde sme testovali funkčnosť komponentov a používateľského rozhrania.



\newpage
\chapter{Analýza problému}

\hspace{15pt} Zaznamenávanie a prehrávanie videa na mobilnom zariadení je možné vzhľadom na rôzne verzie Androidu vykonať rôznymi spôsobmi. V práci sa zameriavame hlavne na prehrávanie zvukového súboru v pozadí pri zaznamenávaní videa alebo v náhľade pred spracovaním videa. 

Pri zaznamenávaní videa je dôležité, aby sa prehrávala zvolená zvuková nahrávka. Zaznamenanie sa ukončí automaticky dokým sa prehrá zvuk v pozadí, alebo používateľ zaznamenávanie ukončí predčasne. Hneď po ukončení sa zobrazí náhľad zaznamenaného videa spolu so zvukovou nahrávkou v pozadí. Samotné nahraté video z kamery neobsahuje žiadnu zvukovú stopu.

\section{Knižnice pre spracovanie videa}

\hspace{15pt} Spracovanie videa v OS Android je možné uskutočniť rôznymi knižnicami. Medzi najviac používané knižnice patria:
\begin{itemize}
\item FFmpeg
\item Mp4parser
\item Media for Mobile
\end{itemize}

\subsection{FFmpeg}

\hspace{15pt} FFmpeg je popredný multimediálny rámec, ktorý dokáže dekódovať, kódovať, muxovať, demuxovať, streamovať, transkódovať, filtrovať a prehrávať. Tieto funkcie sú definované ako možnosti spracovania súboru v príkazovom riadku. Program sa používa na profesionálne spracovanie obrázkov a videí. Podporuje verzie Androidu 4.1 a vyššie, avšak aktuálna verzia knižnice FFmpegAndroid:0.3.2 nie je funkčne stabilná na verzií Androidu 10. 

V programe vieme pracovať s rôznymi vstupmi a so zmenami pomocou možností spracovania vytvoríme nový výstupný zdroj. Každý vstupný a výstupný zdroj môže v princípe obsahovať určitý počet formátov súboru (video, zvuk ,titulky, príloha a dáta). 

Pravidlo je, že ako možnosť spracovania súboru platí iba pre nasledujúci vstupný alebo výstupný súbor a medzi súbormi sa resetujú. Preto je poradie možnosti v príkaze veľmi dôležité. Najskôr je treba zadať vstupné a potom všetky výstupné súbory.
 \cite{ffmpeg01}

\begin{figure}[h]
    \centering
    \includegraphics[width=0.6\textwidth]{images/obrazok1.png}
    \caption{Názorná ukážka ako vyzerá príkaz pomocou Ffmpeg knižnice.}
    \label{fig:obr01}
\end{figure}

\subsection{Práca FFmpeg}

\begin{enumerate}


\item Knižnica \textbf{libavcodec Library} je volaná pre načítanie vstupného súboru. Ktorý demultiplexuje vstupný súbor a poskytuje pakety kódovaných dát.

\item Tieto dekódované dátové pakety pracujú ako vstup pre dekodér. Ktorý, dekóduje dáta a poskytuje nám dekódované dátové rámce. Tieto rámce sú nekomprimované.


\item Komprimované dáta idu cez filtračný proces, ktorý je vykonaný pomocou \textbf{libavfilter Library}. Dáta prechádzajú z rôznych filtrových reťazcov. Následne sa vytvorí filtrový graf.

Filtrový graf sa skladá z \textbf{jednoduchého filtra} a \textbf{zložitého filtra}. Jednoduché filtre poskytujú jeden výstupný tok pre jeden vstupný tok. Zložité filtre majú rozdielny výstup pre vstup. Následne výstup smeruje do kódera.

\item Kóder obdrží spracované dáta z filtra. Následne ich odošle do Multiplexora.

\item Kódované dátové pakety sú následne "multiplexované" na poskytnutie výstupného súboru.

 \cite{ffmpeg02}

\end{enumerate}

\subsection{Nástroje FFmpeg}

\hspace{15pt} Knižnica poskytuje tri základne nástroje na spracovanie.
\textbf{FFmpeg} je vysoko konfigurovateľný. Tento konfiguračný skript môže prijať veľa rôznych argumentov, ktoré ovplyvňujú výstup celého procesu. Najťažšia časť práce FFmpegu v systéme Android je odovzdať správne argumenty.
\textbf{FFplay} poskytuje prenosný médiový prehrávač používajúci FFmpeg a SDL knižnicu. Používa sa ako testovacie miesto pre rôzne FFmpeg API verzie.
\textbf{FFprobe} zhromažďuje informácie z multimediálnych tokov. Používa sa napríklad pre kontrolu formátu použitého kontajnera.

\subsection{Mp4parser}

\hspace{15pt} Knižnica poskytuje Java API na čítanie, zápis a vytváranie MP4 kontajnerov. Manipulácia s kontajnermi sa líši od kódovania a dekódovania videa a zvuku.

Typické úlohy pre Mp4parser:
\begin{itemize}
\item Zlúčenie zvuku a videa do MP4 súboru
\item Pripájanie nahrávok, ktoré používajú rovnaké nastavenia kódovania
\item Pridávanie alebo zmena metadát
\item Skrátenie nahrávok vynechaním rámcov

\end{itemize}

\subsection{Media for Mobile}

\hspace{15pt} Knižnica poskytuje sadu ľahko použiteľných komponentov a rozhranie API pre rôzne typy médií ako je úprava videa a snímanie. Poskytuje určitý počet kompletných potrubí pre prípady použitia. Následne poskytuje možnosť pridať do týchto potrubí vyvinuté používateľom komponenty.

Možnosti pre spracovanie videa: prekódovanie videa, pripojenie videa, vystrihnutie videa, video efekty, zvukové efekty, časové mierky, snímanie videa a hier.

\section{Knižnice pre zaznamenávanie videa}

\hspace{15pt} Pred spracovaním videa je potrebné získať vstupný zdrojový súbor. Jedno z možností je zaznamenávanie videa. Výsledne video s ktorým budeme pracovať bude bez zvuku. Pred zaznamenaním je potrebné nastaviť maximálnu dĺžku výstupného súboru. Preto pre naše účely používame knižnicu CameraView, ktorá obsahuje potrebné metódy pre našu implementáciu.

\subsection{CameraView}

\hspace{15pt} CameraView je vysoko úrovňová knižnica, ktorá sníma obrázky a videa. Poskytuje flexibilitu s ktorou nám pomôže pri riešení rôznych problémov. Poskytuje automatické systémové povolenia.

Kamera informuje o udalostiach z kamery, ktoré sa udejú, a to buď samostatne, alebo po akcii vývojára. Ak chceme získať prístup k týmto udalostiam, musíme nastaviť minimálne jednu inštanciu v listenery pre kameru.

Kamera poskytuje základné udalosti ako napríklad informácia, či je kamera otvorená alebo zatvorená a začiatok a koniec zaznamenávania. Následne po ukončení zaznamenávania je volaná výsledná metóda s parametrom obsahujúci výstup z kamery. Kamera poskytuje aj snímanie obrázkov. Poskytuje filtre v reálnom čase, gestá, vodoznaky, spracovanie rámcov a výstup rôzne definovanej veľkosti. 

Knižnica používa triedu GLSurfaceView pre náhľad z kamery. Trieda poskytuje pomocné triedy na správu kontextu EGL, komunikáciu medzi vláknami a interakciu so životným cyklom aktivity. Napríklad GLSurfaceView vytvorí vlákno na vykreslenie a tam nakonfiguruje EGL kontext. Po pozastavení aktivity sa stav automaticky vyčistí.

\subsection{CameraKit}

\hspace{15pt} Knižnica poskytuje konzistentné výsledky zaznamenávania kamery, prispôsobovacie služby na základe mierky a rôzne možnosti pre zaznamenávanie. Na zaznamenávanie videa a snímanie obrázkov funguje plynule ten istý režim pre náhľad pred snímaním. Knižnica automaticky spracuje systémové povolenia. Podporuje automatické prispôsobovanie ukážky. Vytvorí náhľad kamery používateľom rôzne definovanej veľkosti. Automatické orezanie výstupu, aby sa pevne zhodovala s rozmermi z náhľadu kamery. Poskytuje podporné funkcie ako sú gestá na zaostrenie a priblíženie. Podporuje atribúty potrebné k zaznamenaniu videa ako sú, nastavenie prednej alebo zadnej kamery, blesk, priblíženie, megapixely a efekty v reálnom čase. 

V nastavení verzie beta3.11 je parameter onConfigurationChanged nastavený tak, aby sledoval zmeny veľkosti obrazovky v systéme. Pri otáčaní zariadenia spôsobí neočakávaný výstup v systéme Android verzie 7.0 alebo vyššej.

\cite{CameraKit}

\section{Knižnice pre prehrávanie}

\hspace{15pt} V tejto práci potrebujeme prehrávať video v náhľade pred a po spracovaním videa.  Náhľad video ukážky sa zobrazí hneď po ukončení nahrávania videa. Pre účely našej aplikácie budeme používať knižnicu ExoPlayer, ktorá poskytuje implementačné metódy ako spájanie alebo strihanie médií, zmenu rýchlosti pomocou ktorej vieme prehrávač spomaliť alebo zrýchliť.

\subsection{ExoPlayer}

\hspace{15pt} ExoPlayer je prehrávač médií, ktorý sa používa pre aplikácie v Androide. Prehráva video a zvuk z lokálneho ale aj z internetového zdroja. Má schopnosť aktualizovať prehrávač spolu s aplikáciou. Dokáže prispôsobiť a rozšíriť prehrávač pre náš konkrétny účel použitia. Podporuje zoznam skladieb, ako klipy, zlučovanie a opakovanie prehrávania médií. Podpora pre DASH a SmoothStreaming, ktorá ani jedna z nich nie je podporovaná programom MediaPlayer. Taktiež podpora pokročilých funkcií HLS. Podporuje rôzne média formátov s minimálnou verziou Androidu 4.1.\cite{exoplayer}

Základný komponent MediaSource definuje a zároveň poskytuje média, aké ma prehrávač prehrávať. Táto inštancia by sa mala použiť iba raz, to znamená že by sa nemala ďalej opakovať. Záleží aj na tom, aké dáta chceme prehrávať. Napríklad, ak chceme prehrať mp3 súbor, je treba použiť inštanciu ExtractorMediaSource.

Prehrávač obsahuje komponent pre zoznam skladieb. Vytvoríme osobitné inštancie pre video alebo zvukovú nahrávku. Pomocou ištancie ConcatenatingMediaSource ich dokážeme nahrať do zoznamu a neskôr ich pustiť podľa poradia ako sme zadali. Pomocou implementácie inštancie MergingMediaSource, dokážene spojiť minimálne dve rôzne zdrojové súbory do jedného výstupu. Inštancia ClippingMediaSource poskytuje parametre pre odstrihnutie časového úseku daného média v prehrávači.

ExoPlayer V2 obsahuje niekoľko komponentov používateľského rozhrania, ktoré nie sú súčasťou balenia, najmä:


\begin{itemize}
    \item PlaybackControlView je pohľad na ovládanie inštancií ExoPlayer. Zobrazuje štandardné ovládacie prvky prehrávania vrátane tlačidla prehrávania / pozastavenia, tlačidiel rýchleho posunu dopredu a dozadu a lišty vyhľadávania.
    \item SimpleExoPlayerView je zobrazenie na vysokej úrovni pre prehrávanie médií SimpleExoPlayer. Počas prehrávania zobrazuje video, titulky a obrázky albumov a zobrazuje ovládacie prvky prehrávania pomocou PlaybackControlView.
\end{itemize}

\subsection{MediaPlayer}

\hspace{15pt}Android poskytuje rôzne spôsoby prehrávania videí, hudby alebo streamov. Jeden z týchto spôsobov je prostredníctvom triedy s názvom MediaPlayer. Vie prehrávať média uložené priamo v pamäti zariadenia, zo zdroja vašej aplikácie alebo z dátového toku prichádzajúceho zo sieťového pripojenia. Prehranie médiového súboru pomocou sieťového pripojenia, musí vaša aplikácia vyžadovať prístup k sieti.

\section{Aplikácie na podobnej báze}

\hspace{15pt} Zaznamenávanie alebo spracovanie videa na mobilnom zariadení Android je možné riešiť rôznymi spôsobmi. Niektoré aplikácie poskytujú výber vstupného súboru priamo cez aplikáciu, alebo pomocou prehliadača mobilného úložiska našej mimo aplikácie. Napríklad väčšina aplikácií integruje spracovanie výstupného súboru priamo v používateľskom rozhraní.

V tejto časti opisujeme aplikácie, ktoré obsahujú funkcionality pre importovanie vstupných súborov a ich následne spracovanie. Prieskum aplikácií v OS Android sme uskutočňovali na platforme Google Play. Na prieskum sme použili určité kritéria, ktoré by mala aplikácia spĺňať. Napríklad ako je uľahčovanie výberu vstupného súboru, nahrávanie alebo spracovanie videa intuitívnym spôsobom pre používateľa. Ale aplikácie obsahujúci implementačný spôsob, ktorý by používateľovi bol poskytnutý náhľad videa počas jeho spracovania, nemali alebo boli len čiastočne implementované. Aplikácia s podobnou funkcionalitou ako naša je Add Audio to Video.

\subsection{Video Cutter}

\hspace{15pt} Je používateľsky jednoduchá aplikácia na úpravu videa s rôznymi výkonnými funkciami ako sú strihanie, zlúčenie, konvertovanie videa na mp3 formát a dokáže meniť zvuk v zvolenom súbore. 
Jednoduchosť je jednou z hlavných kľúčových funkcií, na ktoré sa vývoj aplikácie sústredili. \citep{aplikacia2}.

Aplikácia používa na zaznamenávanie videa kameru mimo aplikácie. Po ukončení nahrávania sa video súbor uloží. Následne sa presunie aplikácia do stavu strihania zaznamenaného videa. Hneď po úprave sa video opäť uloží do priečinku vyhradeného pre strihanie. Na konci je možný náhľad výstupného súboru. Okrem zaznamenania videa je možnosť si súbor vybrať z mobilného zariadenia, ktorý chceme orezávať. Aplikácia používa na spracovanie FFMpeg knižnicu.

\begin{figure}[h]
    \centering
    \includegraphics[width=0.72\textwidth]{images/aplikacia2.jpg}
    \caption{Náhľad do aplikácie Video Cutter.}
    \label{fig:obr03}
\end{figure}

\subsection{Add Audio to Video}

\hspace{15pt} Aplikácia obsahuje nástroje pre úpravu alebo zmenu hudby na pozadí pre video. Poskytuje funkcionality ako sú výber časti zvuku a pridanie vybranej časti zvuku do celého videa. Ak je zvuk menší ako video, zvuk sa opakuje. Spojí vybranú časť zvuku so zvukom na pozadí videa. Vieme zmeniť hlasitosť pôvodného zvuku a vybratého zvuku. Vieme pridať vybranú časť zvuku k vybranej časti videa. V inej časti videa sa prehrá originálny zvuk videa. Nakoniec aplikácia poskytuje náhľad spracovaného videa a vieme ho zdieľať na sociálnych sieťach. Na spracovanie videa aplikácia používa FFmpeg knižnicu \citep{aplikacia3}.

\begin{figure}[h]
    \centering
    \includegraphics[width=0.72\textwidth]{images/aplikacia3.jpg}
    \caption{Náhľad do aplikácie Add Audio to Video.}
    \label{fig:obr04}
\end{figure}

\section{Návrh aplikácie}

\hspace{15pt} V našej aplikácií sa pracuje s dostupnými súbormi, ktoré má používateľ na mobilnom zariadení v úložnom priestore. Pri spracovaní môžu nastať iba dve možnosti. Ak používateľ nemá dostupné video v zariadení na spracovanie má možnosť si zaznamenať video priamo v aplikácií.

\hspace{15pt} Pred začiatkom zaznamenania videa, je treba si vybrať hudbu na pozadie. Na výber hudby je spravený zoznam všetkých dostupných skladieb z vnútorného alebo vonkajšieho úložiska telefónu. Výber si dokážeme uskutočniť aj pomocou vyhľadávania podľa názvu skladby. Po potvrdení, sme schopný následne zaznamenať video a zároveň sa nám automatický začne prehrávať hudba na pozadí, ktorú sme si zvolili v zozname. Komponenty pre zaznamenávanie a prehrávanie sú implementované samostatne. Ukončenie zaznamenania videa vieme uskutočniť dvomi spôsobmi: 
\begin{itemize}
    \item Ak zaznamenávanie sa ukončí skôr ako prerevanie hudby, tak zaznamenané video sa spracuje na danú dĺžku ako je hudba, to znamená že sa video bude prehrávať spomalene. V tejto možnosti je potrebné si vypočítať vydelením o koľko krát je video kratšie ako hudba. Hudba na pozadí zostáva zachovaná bez úpravy.
    \item Zaznamenávanie sa ukončí automaticky hneď po tom ako hudba na pozadí sa prestane prehrávať. To znamená, že zaznamenané video nemôže presiahnuť dĺžku hudby na pozadí. V tomto prípade zaznamenané video a hudba na pozadí majú rovnakú dĺžku prehrávania a nie je potrebne žiadné spracovanie.  
\end{itemize}

Pre prípad výberu videa z mobilného zariadenia, nám aplikácia poskytuje vlastnú galériu, v ktorej sú importované všetky dostupné súbory s formátom videa mp4. Následne po výbere videa z galérie, máme možnosť si vstupné video zostrihať a pridať hudbu na pozadie.

Ihneď po ukončení zaznamenávania alebo strihania máme okamžitý náhľad vstupného videa spolu s hudbou v pozadí. Sú implementované dve  komponenty pre prehrávanie. Samostatne pre hudbu na pozadí a zaznamenané video. Následne v náhľade máme dve možnosti:
\begin{itemize}
    \item V náhľade videa máme možnosť sa vráť a zaznamenávanie znovu zopakovať. Pôvodná hudba na pozadí je automaticky pripravená na začiatok zaznamenávania.
    \item Náhľad videa nám poskytuje uloženie spracovaného súboru do úložného priestoru a zároveň počas toho je možné si ukladané video prehrávať priamo v aplikácií.
\end{itemize}

Ukladanie spracovaného súboru do úložného priestoru je uskutočnené v pozadí aplikácie a je možné to sledovať pomocou notifikácie. Následne po ukončení spracovania, máme možnosť si uložené video zobraziť a prehrať pomocou kliknutím na notifikáciu. Video prehrávač vytvoreného videa nám poskytuje zdieľanie súboru na rôzne platformy ako sú sociálne siete alebo Gmail.
Aplikácia poskytuje vlastnú galériu vytvorených videí.

\newpage

\chapter{Opis riešenia}

\section{Použité technické vybavenie pri vývoji}

\begin{table}[H]

Notebook:

\begin{center}
\begin{tabularx}{0.8\textwidth} { 
  | >{\raggedright\arraybackslash}X 
  | >{\raggedright\arraybackslash}X | }
  \hline
 Operačný systém & Windows 10 \\
 \hline
 Procesor & Intel(R) Core(TM) i5-6200U CPU @ 2.30GHz \\
 \hline
 Vývojové prostredie  & Android Studio 3.6.3  \\
  \hline
 Java Verzia  & Verzia 8 (1.8.0-212)  \\
\hline
\end{tabularx}
\caption{Technické vybavenie počítača použitého pri vývoji}
\end{center}
\end{table}

\begin{table}[H]
Smartfón:

\begin{center}
\begin{tabularx}{0.8\textwidth} { 
  | >{\raggedright\arraybackslash}X 
  | >{\raggedright\arraybackslash}X | }
  \hline
 Názov zariadenia & Lenovo Vibe K5 \\
 \hline
 Rozlíšenie displeja & 1080 x 1920 \\
 \hline
 Procesor  & 8 Jadrový 1.8 GHz  \\
 \hline
 Verzia  & Android 6.0  \\
 \hline
 RAM  & 3.00 GB  \\
  \hline
 Úložisko  & 32.00 GB  \\
   \hline
 Zadná kamera  & 13 MP,1080p@30fp,HDR  \\
  \hline
 Predná kamera  & 8 MP,1080p@30fps  \\
\hline
\end{tabularx}
\caption{Technická špecifikácia smartfónu}
\end{center}
\end{table}

\section{Použité technológie}

\hspace{15pt} Na programovanie aplikácie používame objektovo orientovaný programovací jazyk Java. Tento jazyk jasne dominuje na platforme Android, kde sa postupne rozširuje nový jazyk Kotlin. Java patrí k nepoužívanejším a najrozšírenejším programovacím jazykom, kde jeho syntax vychádza z jazykov C a C++. Na vývoj aplikácie pozívame programovacie prostredie Android Studio 3.6.3. Vývojové prostredie poskytuje jednoduché opravovanie chýb, kompilovanie, generovanie a dopĺňanie kódov. Obsahuje rôzne nápomocne komponenty ako je logcat, kde pri spustení aplikácie sa v tom komponente vypisujú hlásenia nápomocné pri testované alebo hľadaní chýb. Platforma poskytuje generovanie výstupného súbora vo formáte .apk. Poskytuje grafické používateľské rozhranie, ktoré umožňuje presúvať vytvorené komponenty a voľbu rozloženia náhľadu medzi grafickým rozhraním a zdrojovým kódom. Popri vývoji používame GitHub platformu pomocou ktorej cez Android Studio sme intuitívnym spôsobom nahrávali zdrojové kódy na platformu. GitHub poskytuje riadenie prístupu a rôzne funkcie spolupráce, ako je sledovanie chýb a požiadavky na funkcie.

\section{Požiadavky aplikácie}

\hspace{15pt} Aby aplikácia správne fungovala je nutné mať OS Android verziu 5.0 a vyššie (API level 21). Pre používanie našej aplikácie sú nutné ďalšie hardvérové požiadavky ako mať aspoň jednu zabudovanú kameru v zariadení a prípadne internetové pripojenie, ktoré slúži na zdieľanie výstupného súboru prostredníctvom napríklad pomocou aplikácie Messenger alebo Gmail. Ďalej je potrebné mať povolenie pre prístup k súborom na mobilnom zariadení, ktoré bude slúžiť pre načítanie všetkých dostupných súborov v mobilnom zariadení do našej galérie. Takisto je treba mať povolenie pre zaznamenávanie videa. Ako vstupné súbory naša aplikácia vyžaduje médiový formát videa mp4 a pre zvukový súbor formát mp3. Pre správnu funkčnosť aplikácie, je nutné, aby na mobilnom zariadení existoval aspoň jeden zvukový súbor formátu mp3.  

\section{Architektúra galérie}
\label{sec:gallery}


\hspace{15pt} V našej aplikácií používame pri spracovaní videa vstupné médiové súbory ako zvuk a video. Kvôli rôznemu počtu obsahu súborov v úložnom priestore telefónu sme sa rozhodli vytvoriť vlastnú galériu. Galéria obsahuje všetky súbory dostupné z vnútorného alebo vonkajšieho úložného priestoru. Galéria je využitá pri výbere zvukovej stopy, videa a pri náhľade všetkých vytvorených videí pomocou našej aplikácie. Na uľahčenie prístupu k súborom sme implementovali funkcionalitu vyhľadávania v galérií. Súbory vyhľadáva podľa názvu súboru. 
\hspace{15pt} Pre spôsob zobrazovania súborov, sme sa rozhodli použiť knižnicu RecycleView. Kontajner poskytuje efektívne zobrazovanie veľkých a posúvateľných údajov udržiavaním obmedzeného počtu položiek zobrazenia. Položky sa menia za behu na základe akcie používateľa alebo sieťových udalostí \cite{recycleView2}.

\begin{figure}[h]
    \centering
    \includegraphics[width=0.9\textwidth]{images/RecyclerView.png}
    \caption{Názorná ukážka vzťahov medzi jednotlivými triedami \cite{recycleView}. }
    \label{fig:obr05}
\end{figure}


Na zobrazenie údajov v zobrazení RecyclerView potrebujeme mať nasledujúce údaje:
\begin{itemize}
    \item Data
    \item RecycleView
    \item Rozloženie pre jednu položku údajov
    \item Správca rozloženia (A Layout manager)
    \item Adapter
    \item ViewHolder
\end{itemize}

\subsection{Dáta}
\label{sec:data}

\hspace{15pt} V našom prípade používame vytvorené dáta lokálne pomocou súborov z úložného priestoru. Objekt \textbf{Položka} reprezentuje jeden súbor a objekt \textbf{Album} reprezentuje obsah daného priečinku v ktorom sa nachádzajú súbory. Na vytvorenie týchto dát používame nasledujúcu štruktúru: 
\begin{table}[H]

Objekt položka (súbor):
\begin{center}
\begin{tabularx}{1\textwidth} { 
  | >{\raggedright\arraybackslash}X 
  | >{\raggedright\arraybackslash}X 
  | >{\raggedright\arraybackslash}X | }
  \hline
 \textbf{Dátovy typ}  & \textbf{Názov} & \textbf{Popis} \\
 \hline
String & title & Názov \\
 \hline
String  & size & Veľkosť  \\
 \hline
Date  & lastModified & Dátum poslednej modifikácie súbora  \\
 \hline
File  & file & Referencia na tento súbor  \\
  \hline
File  & parentFile & Referencia na priečinok v ktorom je súbor  \\
\hline
boolean  & expanded & Informácia či je položka zakliknutá  \\
\hline

\end{tabularx}
\caption{Náhľad štruktúry objektu Položka. }
\end{center}
\end{table}

\begin{table}[H]

Objekt album (priečinok):
\begin{center}
\begin{tabularx}{1\textwidth} { 
  | >{\raggedright\arraybackslash}X 
  | >{\raggedright\arraybackslash}X 
  | >{\raggedright\arraybackslash}X | }
  \hline
 \textbf{Dátovy typ}  & \textbf{Názov} & \textbf{Popis} \\
 \hline
ArrayList$\langle Položka \rangle$ & položky & List, ktorý obsahuje všetky položky (súbory) v priečinku \\
 \hline
String  & title & Názov priečinka  \\
 \hline
File  & file & Referencia na tento subor  \\
  \hline

\end{tabularx}
\caption{Náhľad štruktúry objektu Album. }
\end{center}
\end{table}

\subsection{RecycleView}
\label{sec:recycleView}

\hspace{15pt} Posúvací zoznam, ktorý obsahuje položky zoznamu. Inštancia RecyclerView, ako je definovaná v súbore rozloženia aktivity, slúži ako kontajner pre zobrazenie položiek. Aplikácia RecyclerView uchováva na obrazovke toľko položiek zobrazenia, koľko sa zmestí na obrazovku. Používa obmedzený počet položiek View, ktoré sa opakovane používajú, keď idú mimo obrazovku. Tento spôsob šetrí pamäť a zrýchľuje aktualizáciu položiek zoznamu.

\subsection{Rozloženie pre jednu položku údajov}

\hspace{15pt} Všetky položky zoznamu vyzerajú rovnako, to znamená, že pre všetky z nich môžeme použiť rovnaké rozloženie. Rozvrhnutie položky sa musí vytvoriť oddelene od rozloženia aktivity, aby bolo možné vytvoriť naraz jednu položku prezerania a vyplniť ju údajmi. Súbor obsahuje všetky potrebné údaje pre rozloženie položky. Pomocou týchto údajov sa dajú vytvoriť atribúty ako je názov, dátum vytvorenia a veľkosť súboru.

\begin{figure}[h]
    \centering
    \includegraphics[width=0.9\textwidth]{images/polozka.jpg}
    \caption{Ilustračný náhľad ako vyzerá rozloženie pre jednu položku údajov. }
    \label{fig:obr08}
\end{figure}

\subsection{Adaptér}

\hspace{15pt} V aplikácii RecyclerView adaptér spája údaje so zobrazujúcimi sa položkami. Funguje ako sprostredkovateľ medzi údajmi a zobrazením. Adaptér prijíma alebo obnovuje údaje, vykoná potrebnú prácu na to, aby sa dal zobraziť v zobrazení a umiestni údaje do zobrazenia.

\subsection{ViewHolder}

\hspace{15pt} Poskytuje položku zobrazenia a metadáta o jej mieste v rámci RecyclerView. Každý ViewHolder obsahuje jednu sadu údajov. Adaptér pridá údaje do každého ViewHoldera, aby sa zobrazil správca rozloženia.

\subsection{Implementovanie obrázkov}

\hspace{15pt} Na implementáciu obrázkov reprezentujúcich obsah videa sme použili knižnicu \textbf{Glide v4}. Glide je rýchla a efektívna knižnica na načítanie obrázkov pre Android zameraná na plynulé posúvanie. Knižnica dokáže spracovať obrázok z danej cesty súboru ako je video \cite{glide}.

Obrázky môžu obsahovať rôzne typy veľkosti a vysoké rozlíšenie. Preto bolo potrebné aby sme zaviedli zmenu veľkosti rozlíšenia obrázka aby nedošlo k zbytočnému zaplneniu pamäte. Zmenu veľkosti sme vykonali pomocou možnosti príkazu \textbf{override}, ktorý ma vstupné parametre šírku a výšku nového obrázka. 

Ďalej knižnica poskytuje príkazy ako načítanie štandardného obrázka pre prípad ak vznikne chyba pri načítaní súboru. Pre dekoráciu poskytuje animácie pri načítaní obrázka.

% \begin{figure}[h]
%     \centering
%     \includegraphics[width=0.95\textwidth]{images/glide.png}
%     \caption{Príklad na načítanie obrázkov použitím knižnice Glide. }
%     \label{fig:obr14}
% \end{figure}

\begin{figure}[H]
\begin{lstlisting}[language=Java, caption=Príklad na načítanie obrázkov použitím knižnice Glide.]
 Glide
        .with(context)
        .load(rowItem.getFile())
        .error(R.drawable.broken_image_foreground)
        .override(140,100)
        .transition(DrawableTransitionOptions.withCrossFade(750))
        .into(imageView);
\end{lstlisting}
\end{figure}

\subsection{Opis architektúry programu}

\hspace{15pt} Architektúra programu na vytvorenie galérie sa skladá z rôznych komponentov, ktoré už boli spomenuté. Ako prvé bolo potrebné vytvoriť funkciu, ktorý nám importuje všetky súbory dostupné v úložnom priestore. Táto funkcia má ako vstupné  parametre absolútnu cestu v úložnom priestore a formát súboru, ktorý chceme vyhľadať (.mp3, .mp4). Pomocou tejto funkcie získame dostupné súbory v podobe listu, ktorý obsahuje súbory. Pomocou výstupného listu, vieme vytvoriť objekt typu Položka reprezentujúci súbor. Každý súbor obsahuje metódu (getParentFile). Táto metóda vráti súbor rodiča, to znamená, že poznáme priečinok v ktorom sa daný súbor nachádza. Vďaka tomu sme vytvorili objekt typu Album, ktorý obsahuje list objektov typu Položka a názov albumu (priečinka). 
Galéria sa skladá z týchto tried:

\begin{itemize}
    \item \textbf{FolderRecycleView.java} - trieda s inštanciou RecycleView reprezentujúca náhľad pre priečinky
    \item \textbf{GalleryRecycleView.java} - trieda s inštanciou RecycleView reprezentujúca náhľad pre videa
    \item \textbf{FolderRecycleViewAdapter.java} - trieda Adaptér obsahuje konštruktor, pomocou ktorého vkladáme do adaptéra dôležité dáta ako list objektov typu Album
    \item \textbf{VideoRecycleViewAdapter.java} trieda Adaptér obsahuje konštruktor, kde vkladáme list objektov typu Položka
    \item \textbf{Album.java} - trieda objektu Album reprezentujúci priečinok, obsahujúca list objektu typu Položka
    \item \textbf{RowItem.java} - trieda objektu Položka reprezentujúca súbor
    \item \textbf{FetchFiles.java} - trieda poskytujúca metódy na načítanie súborov z úložiska a následne ich parsovanie na objekty
    \item \textbf{StoragePath.java} - trieda poskytujúca metódu ktorá vráti cestu vnútorného a prípade aj vonkajšieho úložiska
\end{itemize}

V triede \textbf{FolderRecycleView.java} vytvárame Adaptér. Pomocou konštruktora z triedy \textbf{FolderRecycleViewAdapter.java} nasadíme dáta obsahujúce objekty typu Album do Adaptéra. Adapter slúži ako sprostredkovateľ medzi údajmi a zobrazením. Následne každý údaj (priečinok) v Adapteri, obsahuje ďalšie údaje reprezentujúce objekty typu Položka. Následne trieda \textbf{GalleryRecycleView.java} slúži na zobrazenie položiek, ktorý daný priečinok obsahuje. Vyžaduje vstupné dáta ako list objektu Položka, ktoré získame pomocou objektu typu Album. Následne sú tieto dáta implementované do Adaptéra triedy VideoRecycleViewAdapter.java, ktorá zobrazuje náhľad pre položky.

\begin{lstlisting}[language=Java, caption=Náhľad na rekurzívnu funkciu ktorá načitá všetky súbory podľa cesty a typu súboru.]
 private static ArrayList<File> loadFiles(File root, String fileType){
        ArrayList<File> arrayList = new ArrayList<>();
        File[] files = root.listFiles();
        assert files != null;
        if(files.length != 0){
        for (File file : files) {
            if (file.isDirectory()) {
                arrayList.addAll(loadFiles(file, fileType));
            } else {
                if (file.getName().endsWith(fileType)) {
                    arrayList.add(file);
                }
            }
        }
        }
        return arrayList;
    }
\end{lstlisting}


\section{Vzhľad aplikácie}

\hspace{15pt} Dôležitým faktorom našej práce je kladený dôraz na celkový vzhľad aplikácie. Rozhodli sme sa pre jednoduchý a účinný vzhľad pre používanie. Aplikácia využíva dve základne funkcionality pomocou ktorých je možné vykonať spracovanie videa. Na základe toho sme implementovali v hlavnej aktivite FloatButton, ktorý je vytvorený pomocou knižnice \textbf{SpeedDialView}. Následne po rozkríknutí sa animáciou otvoria dve funkcionality, ktoré sme vyššie spomenuli. Aplikácia ma zároveň pevne definovanú tému, ktorej základná farba je fialova a text biely. Ostatné funkcionality ako vyhľadávanie a informácie o aplikácií sme implementovali v hornej lište, ktorá je známa ako \textbf{ActionBar}.

\begin{figure}[H]
    \centering
    \includegraphics[width=0.38\textwidth]{images/main.jpg}
    \caption{Náhľad na možnosti akcie po vykonanom kliku na tlačidlo v hlavnej aktivite. }
    \label{fig:obr07}
\end{figure}

\subsection{Úprava prehrávača}

\hspace{15pt} V tejto podkapitole sa budeme venovať opisu úpravy náhľadu prehrávača. Pre efektívne využitie sme prehrávač prispôsobili na celú obrazovku zariadenia. Prehrávač \textbf{ExoPlayer} poskytuje možnosť úpravy kontrolného pohľadu. Pre maximálne využitie celej obrazovky sme implementovali komponent typu \textbf{ImageView} na vrchnú časť prehrávača. Tieto komponenty reprezentujú tlačidlo, pomocou ktorých je možné vykonať určitú akciu. Tlačidla ako aj ovládacia čast prehrávača sa dokážu zobraziť klikom na obrazovku. Úprava prehrávača bola vykonaná v súbore \textbf{exo-playback-control-view.xml} .

\hspace{15pt} V našej aplikácií boli úpravy vykonané na dvoch miestach. Pri náhľade videa \textbf{VideoPreview.java} a pri prehrávaní vytvoreného videa \textbf{VideoViewer.java}. V komponente prehrávača, boli pridané presne 4 tlačidla vykonávajúce funkcie pre zatvorenie, uloženie, zdieľanie a vymazanie videa. Tieto tlačidlá menia svoj stav na viditeľný a neviditeľný na základe toho, kde je tento komponent využívaný. Pri náhľade videa využívame tlačidla pre zatvorenie a uloženie, ostatné uchovávajú svoj stav pre neviditeľnosť. Prehrávanie výstupného videa poskytuje tri možnosti a to zatvorenie, zdieľanie a vymazanie videa.

\begin{figure}[H]
    \centering
    \includegraphics[width=0.38\textwidth]{images/exo.png}
    \caption{Náhľad na upravené riadiace možnosti prehrávača v režime celej obrazovky. }
    \label{fig:obr09}
\end{figure}


\section{Hlavné implementačné úlohy}

\hspace{15pt} V tejto časti sa venujeme opisu hlavných implementačných úloh, ktoré ma naša aplikácia plniť. Budeme opisovať návrh zaznamenávania videa na základe vybranej hudby na pozadí a strihanie vybraného videa z galérie, ktoré boli medzi našimi prvými implementáciami v našej aplikácií. Ďalej popisujeme návrh náhľadu prehrávaného videa, ktoré bolo zaznamenané alebo spracované strihaním spolu s vybranou hudbou na pozadí. V poslednom rade popisujeme návrh náhľadu videa vytvorený našou aplikáciou.

\begin{figure}[H]
    \centering
    \includegraphics[width=1\textwidth]{images/usecase-diagram.png}
    \caption{Diagram prípadov použitia. }
    \label{fig:obr06}
\end{figure}

Na prechod medzi rôznymi aktivitami používame inštanciu Intent, pomocou ktorej vieme posielať rôzne typy objektov medzi aktivitami. Takisto inštanciu používame pre zdieľanie videa ako poskytnutie daného obsahu.


\begin{lstlisting}[language=Java, label={lst:startActivity}, caption=Náhľad na príklad pre začiatok novej aktivity obsahujúce dané parametre.]
Intent intent = new Intent(CameraActivity.this, VideoPreview.class);
        intent.putExtra("mode",mode);
        intent.putExtra("audioUri",audioUri.toString());
        ActivityOptions options = ActivityOptions.makeSceneTransitionAnimation(this);
        startActivity(intent,options.toBundle());
        
\end{lstlisting}

V \autoref{lst:startActivity} na riadku 1 prvý parameter inštancie reprezentuje aktivitu na ktorej sa práve nachdádzame a druhý parameter reprezentuje aktivitu, ktorú práve chceme začať.
V riadku 2 a 3 nastavujeme dané parametre, ktoré chceme preniesť do ďalšej aktivity.
Riadok 4 poskytuje tvorbu animácie pre prechod medzi aktivitami.

\subsection{Výber videa z galérie}

\hspace{15pt} Ako ďalší spôsob pri spracovaní sme použili výber videa z galérie pomocou našej aplikácie. Všeobecnú implementáciu galérie sme opísali v časti \ref{sec:gallery}. Náhľad galérie sa skladá z komponentu RecycleView, ktorý je spomenutý v časti \ref{sec:recycleView}. Spomínaný komponent je zložený z objektov ako je album a položka. Tieto objekty sú podrobne opísane v časti \ref{sec:data}. Existujú rôzne možnosti na implementáciu galérie. Napríklad implementácia, ktorá importuje všetky dostupné videa z úložného priestoru do jednej galérie. Táto implementácia však nie je efektívna pre používateľa. Pre intuitívny výber videa z galérie, sme sa rozhodli implementovať algoritmus pre náhľad do existujúcich priečinkov, ktoré obsahujú videa pre náš účel. 

Po vybratí videa nám aplikácia poskytuje nepovinnú možnosť strihania videa. Strihanie nám poskytuje možnosť vybrať vlastne definovaný začiatok a koniec určitého úseku videa s presnosťou v desiatkach milisekúnd. Implementovaná je knižnica \textbf{Crystal Range Seekbar} pre nastavenie daného rozsahu orezávania videa. Nutnosťou je aby sme vybrali hudbu na pozadie. Následne je možnosť potvrdenia a v momente uskutočnený presun do náhľadu spracovaného videa.

\begin{lstlisting}[language=Java, label={lst:trimVideo}, caption=Náhľad metódy pomocou ktorej vygenerujeme príkaz na orezanie videa. ]
// volanie metody po kliknuti na tlacidlo
trimVideo(rangeSeekBar.getSelectedMinValue().intValue(),
          rangeSeekBar.getSelectedMaxValue().intValue());
          
private void trimVideo(int startMs, int endMs) throws IOException {
        createVideoFileName();
        duration = (endMs - startMs);
        command = new String[]{"-i", audioUri.getPath(), "-ss", getTime(startMs), "-y", "-i", uri.getPath(),"-t",getTime(duration), "-r","25", "-c:v", "copy", "-c:a", "aac", "-shortest", mVideoFileName};
    }
\end{lstlisting}

V \autoref{lst:trimVideo} metóda obsahuje dve parametre reprezentujúce začiatočný a koncový bod orezávania. Hodnoty obdržíme pomocou inštancie komponentu \textbf{rangeSeekBar}. Na riadku 8 vytvárame príkaz pre orezávanie. Príkaz musí obsahovať základne premenné ako cestu hudby,videa a následne začiatočný a koncový bod orezávania. Ako posledná premenná v príkaze je cesta výsledného súbora.

\subsection{Zaznamenávanie videa}

\hspace{15pt} Dnešné telefóny poskytujú rôzne typy kamier a veľkosť obrazovky. Tieto parametre bolo treba ošetriť tak, aby náhľad a výstup z kamery správne fungoval pre každý typ telefónu. Náhľad kamery obsahuje rôzne funkcie ako je pridanie hudby na pozadie, blesk a otočenie kamery. Prehrávanie hudby na pozadí počas zaznamenávania je dôležitou súčasťou tejto implementačnej úlohy. Zaznamenávania sa môže uskutočniť iba prípade, ak je pridaná hudba na pozadí. Hudbu na pozadie je možné pridať klikom na komponent, ktorý je ním označený. 

Ďalšou dôležitou súčasťou náhľadu kamery bolo navrhnúť a vytvoriť jednoducho ovládateľný náhľad pre bezproblémové použitie. Všetky zabudované komponenty v náhľade kamery sú uložené v priehľadnej lište, vďaka čomu bude náhľad kamery viditeľný na celú obrazovku telefónu. Tlačidlo pre pridanie hudby bolo možné urobiť rôznymi spôsobmi. Dôležité atribúty boli označenie a uloženie tlačidla na správne miesto. Rozhodli sme sa tlačidlo pre pridanie hudby uložiť hneď pod tlačidlo pre zaznamenávanie, vďaka čomu bude lepšia interakcia s používateľom. Pre informačne účely bol implementovaný časovač, ktorý sa spustí pri zaznamenávaní. 

Blesk a otočenie kamery sú nepovinné funkcie. Blesk má dve režimy ako sú zapnutie a vypnutie. Blesk sa zapne hneď ako sa začne zaznamenávanie. Vypnutie a zapnutie je možné aj počas zaznamenávania. Na otočenie kamery je implementovaná otáčajúca sa animácia počas akcie pre lepšiu interakciu s používateľom. 

Pri zaznamenávaní videa je dôležité, aby prehrávanie hudby na pozadí sa začalo v momente zaznamenávania. Tento problém je riešený tak, že sú vytvorené dve rôzne komponenty. Jeden je pre zaznamenávanie pomocou knižnice \textbf{CameraView}, ktorá obsahuje potrebné implementačné metódy pre náš účel. Druhý komponent je pre prehrávanie hudby na pozadí pomocou knižnice ExoPlayer. Zaznamenávanie videa ma nastavanú pevnú dĺžku, ktorá sa nedá prekročiť. Nastavenie je urobené na základe dĺžky hudby na pozadí. To znamená, že zaznamenávanie sa môže ukončiť aj automaticky.

Pre rôzne spôsoby ukončenia zaznamenávania sme implementovali metódy, ktoré nám zistia či ide o automatické ukončenie, keď hudba na pozadí sa celá prehrá alebo či ide o manuálne ukončenie.

\begin{figure}[H]
\path{src\main\java\sk\fei\videoeditor\activities\CameraActivity.java} % equivalent to \url{...}, but more semantic
\lstinputlisting[language=Java, label={lst:onClick}, firstline=365, lastline=380, caption=Príklad metódy zavolanej po kliknuti na tlačidlo zaznamenávania. ]{codes/CameraActivity.java}
\end{figure}


Riadok na čísle 11 v \autoref{lst:onClick} je volaná metóda, ktorá spúšťa celý proces zaznamenávania. Metóda je ošetrená s povolením pre zápisu dát do úložiska. Ak povolenie je zamietnuté, zaznamenávanie sa neuskutoční pokiaľ použivateľ povolenie potvrdí.

Riadok na čísle 7 v \autoref{lst:onClick} kamere nastavujeme maximálnu dĺžku zaznamenávania. Argumentom je metóda \textbf{getAudioDuration()}, ktorá získa dĺžku vybranej hudby zo zoznamu. 

\path{src\main\java\sk\fei\videoeditor\activities\CameraActivity.java} % equivalent to \url{...}, but more semantic
\begin{lstlisting}[language=Java, caption=Príklad metódy pre získanie dĺžky súborového média. , label={lst:getAudioDuration} ]
public int getAudioDuration(){
        MediaPlayer mp  = new MediaPlayer();
        try {
            mp.setDataSource(this, audioUri);
            mp.prepare();
        } catch (IOException e) {
            e.printStackTrace();
        }
        return mp.getDuration();
    }
\end{lstlisting}


\begin{figure}[H]
\path{src\main\java\sk\fei\videoeditor\activities\CameraActivity.java} % equivalent to \url{...}, but more semantic
\lstinputlisting[language=Java, firstline=577, lastline=609, caption=Hlavná metóda na prípravu kamery pred začiatkom zaznamenávania. ,  label={lst:storagePermission} ]{codes/CameraActivity.java}
\end{figure}

Riadok na čísle 13 v \autoref{lst:storagePermission} je volaná metóda, ktorá finálne spúšťa zaznamenávanie a zároveň aj prehrávanie hudby na pozadí. Metóda je ošetrená spôsobom, aby sa neuskutočnilo zaznamenávanie pokiaľ nie je nastavená hudba na pozadí.

\begin{figure}[H]
\path{src\main\java\sk\fei\videoeditor\activities\CameraActivity.java} % equivalent to \url{...}, but more semantic
\lstinputlisting[language=Java, firstline=553, lastline=563, caption=Príklad metódy pre spustenie zaznamenávania spolu a hudbou na pozadí. ,  label={lst:captureVideo} ]{codes/CameraActivity.java}
\end{figure}

\subsection{Výber hudby zo zoznamu}

\hspace{15pt} Výber hudby je uskutočnovaný pri zaznamenávani a strihaní videa. Pomocou príslušného tlačidla pre výber hudby uskutočnujeme začiatok aktivity pre výsledok. 

\path{src\main\java\sk\fei\videoeditor\activities\CameraActivity.java} % equivalent to \url{...}, but more semantic
\lstinputlisting[language=Java, firstline=485, lastline=490, caption=Príklad pre začiatok novej aktivity pre výsledok. ,  label={lst:openAudio} ]{codes/CameraActivity.java}

\path{src\main\java\sk\fei\videoeditor\activities\AudioFileRecycleView.java} % equivalent to \url{...}, but more semantic
\lstinputlisting[language=Java, firstline=216, lastline=223, caption=Metóda ktorá potvrdzuje výsledok aktivity v podobe cesty súboru. ,  label={lst:audioSetResult} ]{codes/AudioFileRecycleView.java}


Metóda \autoref{lst:audioSetResult} sa zavolá po potvrední výberu hudby v zozname. Vstupný parameter \textbf{rowItem} je referencia na danú položku v zozname, na ktorú sme klikli. Na riadku číslo 4, vytvaráme objekt \textbf{Intent}, pomocou ktorej je môžné prenášať rôzne typy objektov cez aktivity. Na riadku číslo 5 nastavujeme objektu argument s parametrami pre názov kľuča a cestu absolútnu cestu vybraného súbora. Aktivita pre výber hudby sa ukončí a jej výsledok vraciame v podobe hudby do rodičovskej aktivity.

\path{src\main\java\sk\fei\videoeditor\activities\CameraActivity.java} % equivalent to \url{...}, but more semantic
\lstinputlisting[language=Java, firstline=729, lastline=737, caption=Metóda ktorá obrží súbor z aktivity zoznamu hudby. ,  label={lst:onActitivtyResult} ]{codes/CameraActivity.java}

\subsection{Náhľad a spracovanie videa}

\path{src\main\java\sk\fei\videoeditor\activities\VideoPreview.java} % equivalent to \url{...}, but more semantic
\lstinputlisting[language=Java, firstline=438, lastline=457, caption=Náhľad pre počiatočne inicialozovanie prehrávača v náhľade videa. ,  label={lst:initializePlayer} ]{codes/VideoPreview.java}

V \autoref{lst:initializePlayer} na riadku voláme metótu \textbf{setMediaSource}, ktorá nastaví prehrávač pre video a hudbu podľa toho aké okolnosti nastali napríklad že video je kratšie alebo rovnakej dĺžky ako hudba alebo či video sme chceli orezať.

\hspace{15pt} Náhľad videa je možné uskutočniť rôznymi spôsobmi. Najčastejšie spôsoby sú náhľady videa až po uložení do zariadenia. Avšak sme sa rozhodli pre efektívnejší spôsob využitia, ktorý je náhľad videa ešte pred uložením do zariadenia. 

Spôsob akým je vykonaný náhľad videa ešte pred jeho uložením je, že boli vytvorené dve nezávisle prehrávače pomocou knižnice \textbf{ExoPlayer} pre video a inštancie \textbf{MediaPlayer} pre zvukovú stopu. Pri náhľade videa po strihaní sme použili inštanciu \textbf{ClippingMediaSource}, ktorú sme následne nastavili do prehrávača. Pri náhľade videa po predčasnom (manuálnom) ukončení zaznamenávania, bolo potrebné výsledné video spracovať na istú dĺžku podľa hudby na pozadí. Inými slovami, video sme spomalili pomocou inštancie \textbf{setPlaybackParameters}, ktorú obsahuje knižnica ExoPlayer. V poslednom prípade kde výsledné video zo zaznamenávania má rovnakú dĺžku ako hudba na pozadí, sme použili pre náhľad videa inštanciu \textbf{MergingMediaSource}. Vďaka ktorej sa dve vstupné zdroje spoja a reprezentujú sa ako celok.


\path{src\main\java\sk\fei\videoeditor\activities\VideoPreview.java} % equivalent to \url{...}, but more semantic
\lstinputlisting[language=Java, firstline=477, lastline=496, caption=Náhľad pre počiatočne inicialozovanie prehrávača v náhľade videa. ,  label={lst:setMediaSource} ]{codes/VideoPreview.java}

V \autoref{lst:setMediaSource} metóda obsahuje podmienky potrebné pre nastavenie prehrávačov pre video a hudbu.


Algoritmus pre ukladanie videa na pozadí sme uskutočnili pomocou knižnice \textbf{WorkManager}. Zadané úlohy majú záruku spustenia, aj keď už je aplikácia ukončená. Inými slovami, WorkManager poskytuje rozhranie API, ktoré je šetrné na batérie a ktoré zahŕňa roky vývoja obmedzení správania systému Android na pozadí. Prostredníctvom \textbf{WorkManager} knižnice a pomocou príkazmi z knižnice \textbf{FFmpeg} je vykonávane ukladanie videa do úložného priestoru zariadenia. Pri jej spustení, je sa vytvorí notifikácia, reprezentujúca stav ukladania videa. Následne po ukončení nám notifikácia zahlási koniec a po kliknutí na ňu sa nám zobrazí prehrávanie uloženého videa.

\begin{figure}[H]
\path{src\main\java\sk\fei\videoeditor\activities\VideoPreview.java} % equivalent to \url{...}, but more semantic
\lstinputlisting[language=Java, firstline=176, lastline=181, caption=Príklad pre implementáciu WorkManagera. ,  label={lst:workManager} ]{codes/VideoPreview.java}
\end{figure}

V \autoref{lst:workManager} na riadku 2 posielame agrumenty pre spracovanie videa napríklad \textbf{príkaz} (command), dĺžka (duration) potrebná pre výpočet percenta ukladania a posledný argument reprezentuje absolútnu cestu výsledneho súbora, ktorý je implentovaný v notifikácií po úspešnom ukončení ukladania.

\path{src\main\java\sk\fei\videoeditor\activities\VideoPreview.java} % equivalent to \url{...}, but more semantic
\lstinputlisting[language=Java, firstline=216, lastline=222, caption=Náhľad ako sú vytvorené príkazy pomocou knižnice FFmpeg. ,  label={lst:commands} ]{codes/VideoPreview.java}

V \autoref{lst:commands} na riadku 5 v premennej \textbf{pts} je vypočítana hodnota, podľa ktorej vieme spracovať dĺžku videa na dĺžku hudby.

\begin{lstlisting}[language=Java, caption=Príklad spustenia vykonávania príkazu a jeho príslušné metódy pomocou knižnice FFmpeg.]
ffmpeg.execute(command,new ExecuteBinaryResponseHandler(){
  @Override
  public void onProgress(String message) {
    // vypocitavany status spracovania
    // max je 100, progres aktualny stav
    mBuilder.setProgress(max, progress, false); }
  @Override
  public void onStart() {
    createNotification()  // inicializacia notifikacie } }
\end{lstlisting}

Ak ešte nebolo spustené ukladanie videa, je možné sa vrátiť a vykonať určité zmeny. Po uložení je celý workflow\footnote{pracovný postup} ukončený.


\section{Vyhodnotenie testov}

\hspace{15pt} V tejto podkapitole venujeme testovaniu pomocou čoho overíme funkcionalitu našej aplikácie. Pre operačný systém Android je nevyhnutné skontrolovať funkcionality na mobilných zariadeniach s rôznymi verziami OS Android. Testovanie sa bude týkať hlavných funkcionalít ako je zaznamenávanie, uloženie videa, výber videa a hudby z galérie ale aj samotného používateľského rozhrania. Testovanie bolo vykonávane manuálne s bežnými používateľmi.

Android Studio poskytuje debugger, ktorý nám umožnil odstraňovanie chýb počas vývoja. Pomocou debuggera máme možnosť si vybrať zariadenie na ladenie aplikácie.

\subsection{Funkčnosť komponentov}

\hspace{15pt} Ako prvé sme vykonali testy na funkčnosť komponentov aplikácie. Test kontroluje funkčnosť komponentu ako je zaznamenávanie, prehrávanie a spracovanie videa. Testovanie sa vykonávalo na bežných zariadeniach v domácnosti.


\begin{table}[H]

\begin{center}
\begin{tabularx}{\textwidth}{
| >{\centering\arraybackslash}X
| >{\centering\arraybackslash}X
| >{\centering\arraybackslash}c
| >{\centering\arraybackslash}c
| >{\centering\arraybackslash}c
| >{\centering\arraybackslash}c|} 
  \hline
 \textbf{Mobilné zariadenie}  & \textbf{RAM} & \textbf{Rozlíšenie} & \textbf{Veľkosť obrazovky} & \textbf{Kamera} \\
 \hline
Lenovo Vibe K5 & 2 GB & 1280x720 & 5.0" & 13 MP/1080p \\
 \hline
  OnePlus X & 3 GB & 1920x1080 & 5.0" & 13MP/1080p \\
 \hline
 Xiaomi Redmi 7 & 3 GB & 2340x1080 & 6.3" & 13MP/1080p \\
 \hline
 Moto g6 & 3 GB & 2160x1080 &  5,7" & 12MP/1080p \\
 \hline
OnePlus 7T & 6 GB & 2400x1080 & 6.55" & 48MP/2160p \\
 \hline
 Huawei P20 Lite & 4 GB & 2220x1080 & 5.84" & 16MP/1080p \\
 \hline
 Huawei P Smart & 3 GB & 2160x1080 & 5.65" & 13MP/1080p  \\
 \hline

\end{tabularx}

\caption{Tabuľka parametrov testovaných zariadení. }
\end{center}
\end{table}

\begin{table}[H]

\begin{center}
\begin{tabularx}{\textwidth}{
| >{\centering\arraybackslash}X
| >{\centering\arraybackslash}X
| >{\centering\arraybackslash}c
| >{\centering\arraybackslash}c
| >{\centering\arraybackslash}c
| >{\centering\arraybackslash}c|} 
  \hline
 \textbf{Mobilné zariadenie}  & \textbf{Verzia OS} & \textbf{Zaznamenávanie} & \textbf{Prehrávanie} & \textbf{Strihanie} & \textbf{Uloženie} \\
 \hline
Lenovo Vibe K5 & 6.0 & OK & OK & OK & OK \\
 \hline
  OnePlus X & 6.0 & OK & OK & OK & OK \\
 \hline
 Xiaomi Redmi 7 & 9.0 & OK & OK & OK & OK \\
 \hline
 Moto g6 & 9.0 & OK & OK & OK & OK \\
 \hline
OnePlus 7T & 10.0 & OK & OK & OK & OK \\
 \hline
 Huawei P20 Lite & 10.0 & OK & OK & OK & OK \\
 \hline
 Huawei P Smart & 9.0 & OK & OK & OK & x \\
 \hline

\end{tabularx}

\caption{Tabuľka testu funkčnosti komponentov. }
\end{center}
\end{table}

Aplikácia je podporovaná od verzie 5.0 a testy sa vykonali na rôznych rozlíšeniach obrazovky vo verziách od 6.0 až po súčasne najnovšiu 10.0 verziu OS.

Funkčnosť rôznych komponentov ako je zaznamenávanie, prehrávanie a strihanie funguje správne na viacerých zariadeniach. Testovanie nám otvorilo viaceré možnosti rozšírenia našej aplikácie.

Avšak existuje zariadenie, ktoré nepodporuje uloženie výstupného videa do zariadenia. Chyba pri uložení videa môže spočívať v rôznych nastaveniach daného zariadenia. 

\subsection{Funkčnosť používateľského rozhrania}

\hspace{15pt} Medzi dôležitými faktormi správnej funkčnosti aplikácie patrí aj používateľské rozhranie. Testovanie je zamerané hlavne na načítavanie prvkov aplikácie, funkčnosť ovládacích prvkov ako sú tlačidlá, animácie a dizajn témy.


\begin{table}[H]

\begin{center}
\begin{tabularx}{\textwidth}{
| >{\centering\arraybackslash}X
| >{\centering\arraybackslash}X
| >{\centering\arraybackslash}c
| >{\centering\arraybackslash}c
| >{\centering\arraybackslash}c
| >{\centering\arraybackslash}c|} 
  \hline
 \textbf{Mobilné zariadenie}  & \textbf{Načítavanie} & \textbf{Tlačidlá} & \textbf{Animácie} & \textbf{Dizajn témy} \\
 \hline
Lenovo Vibe K5 & OK & OK & OK & OK \\
 \hline
  OnePlus X & OK & OK & OK & x \\
 \hline
 Xiaomi Redmi 7 & OK & OK & OK & OK \\
 \hline
 Moto g6 &  OK & OK & OK & OK \\
 \hline
OnePlus 7T &  OK & OK & OK & OK \\
 \hline
 Huawei P20 Lite &  OK & OK & OK & OK \\
 \hline
 Huawei P Smart &  OK & OK & OK & OK  \\
 \hline

\end{tabularx}

\caption{Tabuľka testu funkčnosti použivateľského rozhrania. }
\end{center}
\end{table}

Používateľské rozhranie funguje spoľahlivo na všetkých testovaných zariadeniach. Avšak našiel sa prípad, kde dizajn témy bol neočakávaný na obrazovke orezávania videa kde bola zmenená farba na vrchnej lište. V budúcnosti je treba túto poruchu opraviť. 

Priebežné výsledky testovania počas vývoja na rôznych OS verziách nám umožnili opraviť rôzne chyby na používateľskom rozhraní.

\subsection{Prieskum aplikácie}

\hspace{15pt} Zhotovili sme prieskum pre bežných použivateľov. Prieskum poskytuje 16 otázok a vykonalo ich 9 bežných použivateľov, ktorí vykonali aj samotné testovanie. Otázky sa týkali ohľadom účinnosti, efektívnosti a ovládania aplikácie. Náhľad na otázky je priložený v prílohe.


\begin{figure}[H]
1. \textbf{Bolo ľahké sa zorientovať
v úvodnej obrazovke.}

Úvodná obrazovka by mala byť jasná a výstižná. Preto bolo potrebné zaviesť komponenty ako menu pre začiatok akcie či manipulácia s výslednými videami v úvodnej obrazovke tak, aby sa použivateľ vedeľ zorientovať bez určitých problémov.

Výsledok ukazuje, že použivatelia sa vedeli zorientovať v úvodnej obrazovke bez problémov. Tlačidlo pre menu vytvorenia videa bolo dosť zreteľné a intuitívne pre jeho výber. Výsledne videa v úvodnej obrazovke boli prehľadné. Avšak pre jedného uživateľa bol menší problém pri zorientovaní sa v otvorení menu pre akcie. Môže to byť spôsobené, nedostatočnou výstižnosťou ikony tlačidla.

\begin{tikzpicture}
  \begin{axis}[title  = ,
    xbar,
    y axis line style = { opacity = 0 },
    axis x line       = none,
    tickwidth         = 0pt,
    enlarge y limits  = 0.2,
    enlarge x limits  = 0.02,
    symbolic y coords = { Vôbec nie, Skôr nie,Mierne, Skôr áno, Rozhodne áno},
    nodes near coords,
  ]
  \addplot coordinates { (6,Rozhodne áno)         (1,Mierne)
                         (2,Skôr áno)  (0,Skôr nie) (0,Vôbec nie)};

  \end{axis}

\end{tikzpicture}
\caption{Vyhodnotenie otázky - Bolo ľahké sa zorientovať
v úvodnej obrazovke.}
\end{figure}

\begin{figure}[H]
\textbf{2. Bolo ľahké si vybrať možnosť
pre začiatok vytvárania videa.}

Dôležitý proces aplikácie začiná práve pri výbere spôsobu spracovania videa, kde použivateľ má dve možnosti pri výbere vstupného súboru.

Výsledok ukazuje, že použivatelia si bez problémov vybrala spôsob spracovania.

\begin{tikzpicture}
  \begin{axis}[title  = ,
    xbar,
    y axis line style = { opacity = 0 },
    axis x line       = none,
    tickwidth         = 0pt,
    enlarge y limits  = 0.2,
    enlarge x limits  = 0.02,
    symbolic y coords = { Vôbec nie, Skôr nie,Mierne, Skôr áno, Rozhodne áno},
    nodes near coords,
  ]
  \addplot coordinates { (6,Rozhodne áno)         (0,Mierne)
                         (3,Skôr áno)  (0,Skôr nie) (0,Vôbec nie)};

  \end{axis}
\end{tikzpicture}
\caption{Vyhodnotenie otázky - Bolo ľahké si vybrať možnosť
pre začiatok vytvárania videa.}
\end{figure}

\begin{figure}[H]
 \textbf{3. Náhľad kamery je úhľadný a
jednoduchý pre použitie.}

V tejto práci sme vytvorili svoj vlastný náhľad kamery priamo v aplikácií, kde boli implementované rôzne komponenty ako blest, otáčanie prednej a zadnej kamery, tlačidlo pre vybranie hudby na pozadie. Náhľad kamery sme sa snažili urobiť úhľadným spôsobom a preto sme všetky komponenty zakomponovali do priehľadného menu.


\begin{tikzpicture}
  \begin{axis}[title  =,
    xbar,
    y axis line style = { opacity = 0 },
    axis x line       = none,
    tickwidth         = 0pt,
    enlarge y limits  = 0.2,
    enlarge x limits  = 0.02,
    symbolic y coords = { Vôbec nie, Skôr nie,Mierne, Skôr áno, Rozhodne áno},
    nodes near coords,
  ]
  \addplot coordinates { (6,Rozhodne áno)         (0,Mierne)
                         (3,Skôr áno)  (0,Skôr nie) (0,Vôbec nie)};

  \end{axis}
\end{tikzpicture}
\caption{Vyhodnotenie otázky - Náhľad kamery je úhľadný a
jednoduchý pre použitie.}
\end{figure}



\begin{figure}[H]
\textbf{4. Vedel som sa zorientovať pre
výber hudby na pozadie.}

Touto otázkou sme sledovali to, či bolo tlačidlo alebo popis dostatočne zreteľný na to, aby použivateľ vedel vybrať hudbu na pozadie.

Pre viac ako polovicu uživateľov bola orientácia pri výbere hudby bez problémov. Oznámenie alebo tlačidlo pre výber hudby boli intuitívne pre použivateľa, avšak je treba ešte vylepšiť interakciu, aby použivateľ bol jasne oboznámený pre výber hudby.

\begin{tikzpicture}
  \begin{axis}[title  = ,
    xbar,
    y axis line style = { opacity = 0 },
    axis x line       = none,
    tickwidth         = 0pt,
    enlarge y limits  = 0.2,
    enlarge x limits  = 0.02,
    symbolic y coords = { Vôbec nie, Skôr nie,Mierne, Skôr áno, Rozhodne áno},
    nodes near coords,
  ]
  \addplot coordinates { (5,Rozhodne áno)         (0,Mierne)
                         (4,Skôr áno)  (0,Skôr nie) (0,Vôbec nie)};

  \end{axis}
\end{tikzpicture}
\caption{Vyhodnotenie otázky - Vedel som sa zorientovať pre
výber hudby na pozadie.}
\end{figure}


\begin{figure}[H]
\textbf{5. Zoznam hudby bol prehľadný
a jeho výber bol jednoduchý.}

Keďže pri spracovaní videa je nutné si vybrať hudbu na pozadie, bolo nutné vytvoriť zoznam hudby jednoduchý pre jeho výber. Keďže použivateľ môže mať rôzne množstvo hudby v úložnom priestore zariadenia, bolo potrebné zistiť, či bol zoznam hudby prehľadný a výber jednoduchý.

Výsledky ukazujú, že komponent pre výber hudby splnil svoj účel a pre použivateľov bol daný zoznam hudby jednoduchý.

\begin{tikzpicture}
  \begin{axis}[title  = ,
    xbar,
    y axis line style = { opacity = 0 },
    axis x line       = none,
    tickwidth         = 0pt,
    enlarge y limits  = 0.2,
    enlarge x limits  = 0.02,
    symbolic y coords = { Vôbec nie, Skôr nie,Mierne, Skôr áno, Rozhodne áno},
    nodes near coords,
  ]
  \addplot coordinates { (7,Rozhodne áno)         (0,Mierne)
                         (2,Skôr áno)  (0,Skôr nie) (0,Vôbec nie)};

  \end{axis}
\end{tikzpicture}
\caption{Vyhodnotenie otázky - Zoznam hudby bol prehľadný
a jeho výber bol jednoduchý.}
\end{figure}


\begin{figure}[H]
\textbf{6. Bolo ľahké a nápomocné
zaznamenať video spolu s hudbou na pozadí.}

Touto otázkou sme sledovali, či zaznamenanie videa spolu s hudbou na pozadí bol účinný ale aj celý proces zaznamenávania kde zahŕňame náhľad kamery a výber hudby. 

Výsledky použivateľov ukazujú, že manipulácia s hudbou popri zaznamenávanie bola jednoduchá. Použivatelia vedeli zaznamenať video spolu s hudbou na pozadí, avšak nebola dostatočne zreteľná požiadavka pre výber hudby.

\begin{tikzpicture}
  \begin{axis}[title  = ,
    xbar,
    y axis line style = { opacity = 0 },
    axis x line       = none,
    tickwidth         = 0pt,
    enlarge y limits  = 0.2,
    enlarge x limits  = 0.02,
    symbolic y coords = { Vôbec nie, Skôr nie,Mierne, Skôr áno, Rozhodne áno},
    nodes near coords,
  ]
  \addplot coordinates { (4,Rozhodne áno)         (1,Mierne)
                         (4,Skôr áno)  (0,Skôr nie) (0,Vôbec nie)};

  \end{axis}
\end{tikzpicture}
\caption{Vyhodnotenie otázky - Bolo ľahké a nápomocné
zaznamenať video spolu s hudbou na pozadí.}
\end{figure}


\begin{figure}[H]
\textbf{7. Výber videa z galérie bol
jednoduchý a prehľadný.}

Použivateľ môže mať rôzny počet videí v úložnom priestore zariadenia. Bolo potrebné otestovať spôsob implementácie galérie poskytujúci priečinky v ktorých sú videa.

Spôsob implementovania výberu videa z galérie splnil svoj účel. Galéria je jednoduchá a plynulá aj pri väčšom množstve videí z úložného priestoru. Výber videa dopomohla hlavne implementácia priečinkov, pomocou ktorých použivateľ jasne indentifikoval kde sa daný súbor môže nachádzať.

\begin{tikzpicture}
  \begin{axis}[title  = ,
    xbar,
    y axis line style = { opacity = 0 },
    axis x line       = none,
    tickwidth         = 0pt,
    enlarge y limits  = 0.2,
    enlarge x limits  = 0.02,
    symbolic y coords = { Vôbec nie, Skôr nie,Mierne, Skôr áno, Rozhodne áno},
    nodes near coords,
  ]
  \addplot coordinates { (8,Rozhodne áno)         (0,Mierne)
                         (1,Skôr áno)  (0,Skôr nie) (0,Vôbec nie)};

  \end{axis}
\end{tikzpicture}
\caption{Vyhodnotenie otázky - Výber videa z galérie bol
jednoduchý a prehľadný.}
\end{figure}


\begin{figure}[H]
\textbf{8. Spôsob orezania videa bol jednoduchý pre ovládanie.}
Existujú rozné spôsoby ako implementovať orezávanie videa. My sme na to použili knižnicu, pomocou ktorej vieme určiť úsek nového začiatku a konca videa. Chceli sme zistiť, ako naša implementácia funguje v praxi pri určení daného úseku.

Väčšina použivateľov rozhodne súhlasila akým spôsobom je možné orezávať video, avšak použivatelia by prijali možnosť orezávania pri otočení obrazovky na ležato. Týmto spôsobom by bolo oveľa jednoduchšie orezávanie videa pre použivateľov.

\begin{tikzpicture}
  \begin{axis}[title  = ,
    xbar,
    y axis line style = { opacity = 0 },
    axis x line       = none,
    tickwidth         = 0pt,
    enlarge y limits  = 0.2,
    enlarge x limits  = 0.02,
    symbolic y coords = { Vôbec nie, Skôr nie,Mierne, Skôr áno, Rozhodne áno},
    nodes near coords,
  ]
  \addplot coordinates { (6,Rozhodne áno)         (1,Mierne)
                         (2,Skôr áno)  (0,Skôr nie) (0,Vôbec nie)};

  \end{axis}
\end{tikzpicture}
\caption{Vyhodnotenie otázky - Spôsob orezania videa bol jednoduchý pre ovládanie.}
\end{figure}


\begin{figure}[H]
\textbf{9. Náhľad videa spracovaného pred samotným uložením vnímam ako prínos. }

Medzi najdôležitejšou funkcionalitou našej práce je náhľad videa. Preto sme chceli zistiť, ako je vnímaná naša funkcionalita alebo či mala pre použivateľov určitý prínos pri spracovaní videa.

Použivateľia ocenili funkcionalitu náhľadu videa v momente po spracovaní. Avšak samotný náhĺad možno nie je dostatnčne zreteľný pre uloženie či návrat na predošlú obrazovku. 
\begin{tikzpicture}
  \begin{axis}[title  = ,
    xbar,
    y axis line style = { opacity = 0 },
    axis x line       = none,
    tickwidth         = 0pt,
    enlarge y limits  = 0.2,
    enlarge x limits  = 0.02,
    symbolic y coords = { Vôbec nie, Skôr nie,Mierne, Skôr áno, Rozhodne áno},
    nodes near coords,
  ]
  \addplot coordinates { (4,Rozhodne áno)         (1,Mierne)
                         (4,Skôr áno)  (0,Skôr nie) (0,Vôbec nie)};

  \end{axis}
\end{tikzpicture}
\caption{Vyhodnotenie otázky - Náhľad videa spracovaného pred samotným uložením vnímam ako prínos.}
\end{figure}


\begin{figure}[H]
\textbf{10. Prechod medzi obrazovkami
bol jasný a plynulý.}

Mimo funkčnosti samotných komponentov bolo potrebné otestovať aj samotné správanie aplikácie. V tomto bode sme testovali hlavne funkčnosť animácií v prechode medzi obrazovkami. Taktiež sme testovali či bol prechod medzi obrazovkami dostatočne zreteľný pre použivateľa.

Prechod mezdi obrazovkami bol dostatočne zreteľný pre použivateľa. Pri tomto bode treba zohľadiť aj výkonnosť mobilného zariadenia, kde animácia a prechod nemuseli byť jasné a plynulé.

\begin{tikzpicture}
  \begin{axis}[title  = ,
    xbar,
    y axis line style = { opacity = 0 },
    axis x line       = none,
    tickwidth         = 0pt,
    enlarge y limits  = 0.2,
    enlarge x limits  = 0.02,
    symbolic y coords = { Vôbec nie, Skôr nie,Mierne, Skôr áno, Rozhodne áno},
    nodes near coords,
  ]
  \addplot coordinates { (6,Rozhodne áno)         (1,Mierne)
                         (2,Skôr áno)  (0,Skôr nie) (0,Vôbec nie)};

  \end{axis}
\end{tikzpicture}
\caption{Vyhodnotenie otázky - Prechod medzi obrazovkami
bol jasný a plynulý.}
\end{figure}


\begin{figure}[H]
\textbf{11. Proces uloženia videa na pozadí aplikácie vnímam ako prínos.}

Patrí to medzi najdôležitejšou funkcionalitou našej aplikácie. Proces uloženia videa na pozadí poskytuje použivateľovi flexibilitu používania a hlavne táto funkcionalita je šetrná pre batériu.

Väčšina použivateľov je nad mieru spokojná ohľadom prínosu, avšak je ešte potrebné zlepšiť, aby proces ukladania videa netrval veľmi dlho. Následne pri začiatku ukladania videa do zariadenia spustenie notifikácie nebolo dostatočne zreteľné.

\begin{tikzpicture}
  \begin{axis}[title  = ,
    xbar,
    y axis line style = { opacity = 0 },
    axis x line       = none,
    tickwidth         = 0pt,
    enlarge y limits  = 0.2,
    enlarge x limits  = 0.02,
    symbolic y coords = { Vôbec nie, Skôr nie,Mierne, Skôr áno, Rozhodne áno},
    nodes near coords,
  ]
  \addplot coordinates { (5,Rozhodne áno)         (0,Mierne)
                         (4,Skôr áno)  (0,Skôr nie) (0,Vôbec nie)};

  \end{axis}
\end{tikzpicture}
\caption{Vyhodnotenie otázky - Proces uloženia videa na pozadí aplikácie vnímam ako prínos.}
\end{figure}


\begin{figure}[H]
\textbf{12. Bolo účinné prehrávanie a náhľad videa na celej obrazovke.}

Aplikáciu sme sa snažili vypracovať efektívne, čo znamená, aby sa aj obrazovka využila efektívne. Rozhodli sme sa maximalizovať obrazovku pre prehrávanie a tak sme chceli zistiť, či vďaka tomu bolo prehrávanie učinné.

Spôsob maximalizovania obrazovky účel pre používateľov splnil. Taktiež bolo prehrávaniu umožnené aj otáčanie obrazovky. Použivatelia boli spokojný so spôsobom ako sú implementované tlačidlá v rámci prehrávača na celej obrazovke.

\begin{tikzpicture}
  \begin{axis}[title  = ,
    xbar,
    y axis line style = { opacity = 0 },
    axis x line       = none,
    tickwidth         = 0pt,
    enlarge y limits  = 0.2,
    enlarge x limits  = 0.02,
    symbolic y coords = { Vôbec nie, Skôr nie,Mierne, Skôr áno, Rozhodne áno},
    nodes near coords,
  ]
  \addplot coordinates { (5,Rozhodne áno)         (0,Mierne)
                         (4,Skôr áno)  (0,Skôr nie) (0,Vôbec nie)};

  \end{axis}
\end{tikzpicture}
\caption{Vyhodnotenie otázky - Bolo účinné prehrávanie a náhľad videa na celej obrazovke. }
\end{figure}


\begin{figure}[H]
\textbf{13. Navigácia a používanie aplikácie bolo intuitívne.}

Navigácia a aplikácie aplikácie znamená, či rozloženie komponentov na obrazovke a popis k nim bol dostatočne intuitívny na celkové používanie aplikácie.

Väčšina použivateľov nemala žiaden problem s používaním. Navigácia a používanie je intuitívne, avšak treba zohľadniť aj skúsenosti použivateľov s mobilnými technológiami.

\begin{tikzpicture}
  \begin{axis}[title  = ,
    xbar,
    y axis line style = { opacity = 0 },
    axis x line       = none,
    tickwidth         = 0pt,
    enlarge y limits  = 0.2,
    enlarge x limits  = 0.02,
    symbolic y coords = { Vôbec nie, Skôr nie,Mierne, Skôr áno, Rozhodne áno},
    nodes near coords,
  ]
  \addplot coordinates { (5,Rozhodne áno)         (1,Mierne)
                         (3,Skôr áno)  (0,Skôr nie) (0,Vôbec nie)};

  \end{axis}
\end{tikzpicture}
\caption{Vyhodnotenie otázky - Navigácia a používanie aplikácie bolo intuitívne.}
\end{figure}


\begin{figure}[H]
 \textbf{14. Naučil som sa základne funkcie ako spracovať video počas tohto testovania.}
 
 Otázkou sme sledovali hlavne to, akým spôsobom sme implementovali celý proces spracovania videa. Celý proces by nemal byť zložitý pre používanie, čo znamená, že základné funkcie by nemali byť zložité pre ich používanie.
 
 Výsledky ukazujú jednoduchosť základných funkcií pri ktorých bolo zreteľné čo sa od nich očakáva. Avšak ešte existujú možnosti ako zjednodušiť základne funkcie ako napríklad umožnenie použivateľovi o menej klikov naviac na obrazovke.

\begin{tikzpicture}
  \begin{axis}[title  =,
    xbar,
    y axis line style = { opacity = 0 },
    axis x line       = none,
    tickwidth         = 0pt,
    enlarge y limits  = 0.2,
    enlarge x limits  = 0.02,
    symbolic y coords = { Vôbec nie, Skôr nie,Mierne, Skôr áno, Rozhodne áno},
    nodes near coords,
  ]
  \addplot coordinates { (4,Rozhodne áno)         (0,Mierne)
                         (5,Skôr áno)  (0,Skôr nie) (0,Vôbec nie)};

  \end{axis}
\end{tikzpicture}
\caption{Vyhodnotenie otázky - Naučil som sa základne funkcie ako spracovať video počas tohto testovania.}
\end{figure}


\begin{figure}[H]
\textbf{15. Funkcie na obrazovkách boli jasne a zreteľné.}
 
Funkcie ako tlačidlo či ikona by mali byť dostatočne jasne a zreteľné na spôsob ich využívania. Funkcie na obrazovke by mali jasne reprezentovať to, čo sa od nich očakáva.

Použivatelia zhodnotili funkcie na obrazovke pozitívne. Ikony a popis k nim je dostatočne výstižný ich funkcionalite. Avšak určité funkcie možno neboli dostatočne výstižné napríklad pri uložení videa alebo hodnoty časov pri orezávani videa, kde pri hodnotách nie je uvedený popis.
 
\begin{tikzpicture}
  \begin{axis}[title  =,
    xbar,
    y axis line style = { opacity = 0 },
    axis x line       = none,
    tickwidth         = 0pt,
    enlarge y limits  = 0.2,
    enlarge x limits  = 0.02,
    symbolic y coords = { Vôbec nie, Skôr nie,Mierne, Skôr áno, Rozhodne áno},
    nodes near coords,
  ]
  \addplot coordinates { (5,Rozhodne áno)         (1,Mierne)
                         (3,Skôr áno)  (0,Skôr nie) (0,Vôbec nie)};

  \end{axis}
\end{tikzpicture}
\caption{Vyhodnotenie otázky - Funkcie na obrazovkách boli jasne a zreteľné.}
\end{figure}


\begin{figure}[H]
\textbf{16. Aplikácia na efektívnu úpravu videa bola užitočná.}

Otázka bola mierená na výsledok aplikácie. Keďže našou úlohou tejto práce bolo vytvoriť aplikáciu na efektívnu úpravu videa, chceli sme zistiť, či naša aplikácia splnila použivateľov účel. 

Na výsledku tejto otázky môžeme vidieť, že cieľ našej práce sme splnili. Existuje ešte množnstvo spôsobov ako je možné túto aplikáciu vylepšiť. Použivatelia zhodnotili, že aplikácia bola užitočná.

\begin{tikzpicture}
  \begin{axis}[title  = ,
    xbar,
    y axis line style = { opacity = 0 },
    axis x line       = none,
    tickwidth         = 0pt,
    enlarge y limits  = 0.1,
    enlarge x limits  = 0.02,
    symbolic y coords = { Vôbec nie, Skôr nie,Mierne, Skôr áno, Rozhodne áno},
    nodes near coords,
  ]
  \addplot coordinates { (3,Rozhodne áno)         (0,Mierne)
                         (6,Skôr áno)  (0,Skôr nie) (0,Vôbec nie)};

  \end{axis}
  
\end{tikzpicture}
\caption{Vyhodnotenie otázky - Aplikácia na efektívnu úpravu videa bola užitočná.}
\end{figure}


\begin{table}[H]
\centering
\begin{tabularx}{\textwidth}{
| >{\centering\arraybackslash}X
| >{\centering\arraybackslash}X
| >{\centering\arraybackslash}X |} 
\hline
\textbf{Odpovede} & \textbf{Počet výskytov} & \textbf{Percento} \\ 
\hline
Rozhodne áno                                    &               85                      &       59\%                        \\ \hline
Skôr áno                                        &                52                     &        36,1\%                       \\ \hline
Mierne                                          &                5                     &          3,5\%                     \\ \hline
Skôr nie                                        &                 2                    &          1,4\%                     \\ \hline
Vôbec nie                                       &                 0                    &            0\%                   \\ \hline
Celkovo                                         &                 144                    &          100\%                     \\ \hline


\end{tabularx}

\caption{Náhľad na vyhodnotenie prieskumu aplikácie. }
\end{table}

Treba zohľadniť skúsenosti použivateľa s aplikáciami ale aj výkonnosť zariadenia na ktorých používatelia vykonávali testy. Prieskum aplikácie možeme zhodnotiť úspešne. Použivatelia sa rýchlo naučili funkcionality, ktoré ponúka naša aplikácia. Zaznamenávanie či výber hudby bol pre nich jednoduchý a intuitívny. Náhľad videa spoločne s jeho uložením je prínosom pre účel testovatých použivateľov.

Avšak, dostali sme aj menej pozitívne hodnotenie. Ako napríklad strihanie videa, kde nie je možné otáčanie obrazovky. Určité tlačidla alebo popis k nim neboli dostatočné zreteľné, avšak je treba zohľadniť jazykovú bariéru, keďže naša aplikácia je vypracovaná v cudzom jazyku a nie každý ovláda anglický jazyk.





% \begin{figure}[H]
% \begin{tikzpicture}
% \begin{axis}[
%     ybar,
%     enlargelimits=0.15,
%       percentage plot,
%     ylabel=Odpovede v percentach,
%     bar width=1.75cm,
%     symbolic x coords={Rozhodne ano,Skor ano,Mierne,Skor nie, Vobec nie},
%     xtick=data
%     ]
% \addplot table [percentage series=1] {\data};
% \legend{1. seria}
% \end{axis}
% \end{tikzpicture}
% \caption{Graf vyhodnotenia prieskumu aplikácie.}
% \label{prieskum}
% \end{figure}

% \autoref{prieskum} obsahuje zozbierané výsledky v percentách, kde odpovede Rozhodne áno a Skôr áno (Súhlas) mali dokopy 94.5\% , zatiaľ čo odpovede Mierne, Skôr nie a Vôbec nie (Nesúhlas) mali dokopy 5.5\% .



\newpage

\chapter*{Záver}
\addcontentsline{toc}{chapter}{Záver}

\hspace{15pt} Vytvoriť aplikáciu pre efektívne spracovanie videa v OS Android, to bolo cieľom tejto bakalárskej práce. V prvej kapitole sme rozobrali a porovnali možné knižnice použiteľné v našej problematike. Porovnali sme aplikácie, ktoré sa zaoberajú podobnej problematike a následne sme urobili návrh nášho riešenia. V druhej kapitole sme sa venovali opisu riešenia v ktorom sú zahrnuté podkapitoly opisujúce návrh a implementáciu rôznych komponentov nevyhnutné pre tvorbu našej aplikácie. Následne sme vyhodnotili uskutočnené testy, na základe ktorých sme skontrolovali funkčnosť implementovaných komponentov.

Naša aplikácia ponúka spracovanie videa spolu s hudbou na pozadí. Aplikácia ponúka spracovanie na základe zaznamenávania videa alebo výberu hudby z úložného priestoru zariadenia. Zaznamenávanie je prispôsobené pre rôzne veľkosti obrazovky. Následne poskytuje funkcionalitu pre výber hudby na pozadie pre obe možnosti spracovania. Výber je vyhotovený, aby používateľ nebol zaťažený rôznymi krokmi navyše. Medzi najdôležitejšie funkcionality patrí aj možnosť si prehrávať potencionálne výsledne video v náhľade, kde sa dá vykonať ukladanie videa do zariadenia a to všetko na pozadí aplikácie čo šetrí aj batériu. Ukladanie videa neobmedzuje používateľa v možnosti začať ďalší proces spracovania.

Všetky jednotlivé body zadania tejto práce boli úspešné vykonané a náš očakávaný cieľ práce bol splnený. Spracovanie videa intuitívnym spôsobom pre ovládanie, ktoré šetrí batériu a je prispôsobené na rôzne veľkosti obrazoviek s okamžitou možnosťou náhľadu videa počas jeho spracovania sa v pozadí aplikácie. Testovanie na rôznych zariadeniach potvrdilo funkčnosť všetkých komponentov, avšak popri testovaním sa našli niektoré zariadenia s OS Android 9, kde nefunguje spracovanie výsledného videa. Skúmaním sme zistili, že systémové zariadenie zamietne prístup pre vykonanie procesu.  Naša aplikácia bola implementovaná tak, aby sa dala v budúcnosti rozšíriť pridaním rôznych komponentov. Napríklad pri výbere hudby zo zoznamu by používateľ mohol nielen prehrávať hudbu pred jeho pridaním ale mohol by hudbu orezať s požadovaným rozsahom. Existujú rôzne iné možnosti spracovania ako spájanie vstupných súborov alebo pridávanie efektov do videa ako text a obrázky. Tieto vylepšenia môžeme uskutočniť ako možné nasledujúce zadanie tejto práce.


\newpage

\bibliographystyle{plain}
\bibliography{references}

\newpage

\makeatletter
\pagenumbering{roman}

\listofappendices
\newpage
\appendix

\chapter{Používateľský manuál}

\hspace{15pt} Aplikácia Audio Video Mixer vykonáva funkcionality na rôznych obrazovkách obsahujúce dané kroky, vďaka ktorým efektívne spracujeme video. V tomto používateľskom manuáli opíšeme dané spôsoby, ako sa správnym postupom naša aplikácia používa.

\begin{figure}[H]

    \hspace{15pt} \textbf{Úvodná obrazovka}
    
\begin{itemize}
        \item Úvodná obrazovka obsahuje zoznam vytvorených videí v tejto aplikácií, ktorý je prázdny ak nebolo zatiaľ nič vytvorené. 
        \item V pravo dole je umiestnené tlačidlo pomocou ktorého otvoríme menu v ktorom sú možnosti spracovania videa.
        \item Pridržaním na položku sa nám ovládací panel zmení na režim, ktorý poskytuje vymazanie súborov.
    \end{itemize}

   
  
\vspace{15pt}


\hspace{0.5cm}
  \includegraphics[width=0.95\textwidth]{images/uvodna0.jpg}
    \caption{Podrobný opis úvodnej obrazovke a jej správania.}
    \label{fig:obr10}
    
\end{figure}

\begin{figure}[H]
\begin{minipage}[b]{0.55\linewidth}

    \hspace{15pt} \textbf{Zaznamenanie videa}

    \begin{itemize}
        \item V náhľade kamery sú zabudované možnosti ovládania pre otočenie pre zadnú a prednú kameru.
        \item Zadná kamera poskytuje možnosť zapnúť a vypnúť blesk.
        \item Po kliknutí na ikonu znázornenú pod názvom \textbf{Select audio} sa vykoná výber hudby na pozadie z galérie.
        \item Ak hudba bola vybratá, text v ikone sa zmení na \textbf{Audio Selected} a stále je možné si hudbu zmeniť.
    \end{itemize}

   

    \hspace{15pt} 
    
\end{minipage}
\hspace{0.5cm}
\begin{minipage}[b]{0.35\linewidth}
  \includegraphics[width=1\textwidth]{images/rec.png}
    \caption{Detailný náhľad na ovládanie pre zaznamenávanie videa. }
    \label{fig:obr11}
\end{minipage}


\begin{minipage}[b]{0.55\linewidth}
  \includegraphics[width=1\textwidth]{images/audio.png}
    \caption{Náhľad pri výbere hudby. }
    \label{fig:obr12}
\end{minipage}
\hspace{0.5cm}
\begin{minipage}[b]{0.4\linewidth}

 \vspace{40pt} 

    \hspace{15pt} \textbf{Výber hudby}

 \begin{itemize}
        \item Táto obrazovka nám poskytuje náhľad na všetky dostupné skladby v úložnom priestore.
        \item Kliknutím na danú položku sa nám otvorí možnosť prehrávania a následne je možné skladbu pridať potvrdením pomocou príslušného tlačidla.
       
    \end{itemize}
    
\end{minipage}
\end{figure}


\begin{figure}[H]

    \hspace{15pt} \textbf{Výber videa z galérie}

 \begin{itemize}
        \item  V tejto časti máme možnosti si vybrané video z galérie zostrihať. Guličkou označené ikony slúžia na nastavenie nového začiatku a konca videa. 
        \item Tlačidlo  \includegraphics[width=0.16\textwidth]{images/audio plus.png} slúži na pridanie hudby do pozadia videa
        
        \item Následne po pridaní hudby, pomocou tlačidla \includegraphics[width=0.15\textwidth]{images/edit.jpg} môžeme hudbu zmeniť.
        
        \item Čas v strede pri strihaní reprezentuje novú dĺžku videa.
        
        \item Ikona \includegraphics[width=0.06\textwidth]{images/fajka.jpg} sa zobrazí po pridaní hudby na pozadie, pomocou ktorej potvrdíme ukončenie strihania.
     
    \end{itemize} 
   
    \vspace{25pt}
    
  \includegraphics[width=0.95\textwidth]{images/trim.jpg}
    \caption{Náhľad na kroky od výberu z galérie po strihanie videa. }
    \label{fig:obr13}
\end{figure}


\begin{figure}[H]

    \hspace{15pt} \textbf{Náhľad videa pred uložením}

\vspace{10pt}
\hspace{15pt} Prehrávač pri náhľade videa obsahuje nasledovné funkcie: 
 \begin{itemize}
        \item Pomocou ikony \includegraphics[width=0.05\textwidth]{images/close.png} sa vieme vrátiť naspať, alebo ak video bolo uložené, zavrieme a vrátime na domovskú obrazovku.
        \item Ikona \includegraphics[width=0.05\textwidth]{images/save.png} slúži na uloženie videa do úložného priestoru.
        \item Po uložení videa sa nám zobrazí notifikácia pomocou ktorej spustíme video v prehrávači.
        
    \end{itemize} 
   
\end{figure}

\begin{figure}[H]

    \hspace{15pt} \textbf{Prehrávač vytvoreného videa}
    
    \vspace{10pt}
    
\hspace{15pt} Prehrávač videa obsahuje nasledovné funkcie: 
 \begin{itemize}
        \item Ikona \includegraphics[width=0.05\textwidth]{images/close.png} prehrávač zatvorí a vratime sa na domovskú obrazovku.
        \item Ikona \includegraphics[width=0.05\textwidth]{images/share.png} slúži pre zdieľanie videa na sociálne siete.
        \item Ikona \includegraphics[width=0.05\textwidth]{images/delete.png} video vymaže z úložného priestoru.
        
    \end{itemize}

   
\end{figure}

\newpage

\begin{minipage}{\textwidth}

\chapter{Dotazník}

\vspace*{-1.24in}
        \centerline{\includegraphics[scale=0.85]{includes/Otázky_priloha.pdf}}
\vspace{\floatsep}
\end{minipage}


\end{document}
